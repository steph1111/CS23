\documentclass[11pt, letterpaper, includehead]{article}

%%%%%%%%%%%%%%%%%%%%% Pre-document %%%%%%%%%%%%%%%%%%%%%
\usepackage{fancyhdr}
\usepackage{float}
\usepackage{array}
\usepackage{nicematrix}
\usepackage{enumitem}
\usepackage{titlesec}
\usepackage{multicol}
\usepackage{scrextend}
\usepackage{hyperref}
\usepackage{amssymb}
\usepackage{amsthm}

\setlength{\parindent}{0pt} % Remove auto paragraph indents

% Get rid of those big margins
\usepackage[margin=1in]{geometry}
\newlength\titleindent
\setlength\titleindent{2cm}

\makeatletter
\def\@mathmargin{1in}
\makeatother

\titleformat{\section}[runin]
    {\normalfont\bfseries}% formatting commands to apply to the whole heading
    {\thesubsection}% the label and number
    {0.5em}% space between label/number and subsection title
    {}% formatting commands applied just to subsection title
    []% punctuation or other commands following subsection title

\theoremstyle{plain}

\newtheoremstyle{mydefinition}  % Name of the style
  {} % Space above
  {} % Space below
  {\normalfont} % Body font (normal, not bold or italic)
  {} % Indent amount
  {\itshape} % Theorem head font (italic)
  {.} % Punctuation after theorem head
  {.5em} % Space after theorem head
  {} % Theorem head spec (can be left empty)

\theoremstyle{mydefinition}
\newtheorem{defin}{Definition}

\newtheoremstyle{myproperty}  % Name of the style
  {} % Space above
  {} % Space below
  {\normalfont} % Body font (normal, not bold or italic)
  {} % Indent amount
  {\itshape} % Theorem head font (italic)
  {.} % Punctuation after theorem head
  {.5em} % Space after theorem head
  {} % Theorem head spec (can be left empty)

\theoremstyle{myproperty}
\newtheorem{prop}{Property}

\begin{document} 

\pagestyle{fancy}
% Header
\fancyhead{}
\fancyhead[L]{\textbf{CS23:} Assignment \#8}
\fancyhead[R]{\href{mailto:stepanielh1111@gmail.com}{L'Heureux} \thepage}
% No page numbers for footer
\fancyfoot{}


\begin{center}
    \Large{\textbf{Assignment 8}}\\
    \Large{Probability}
\end{center}

\begin{enumerate}[label=\textbf{\arabic*}., leftmargin=*]
\item Two faces of a fair six-sided die are painted red, two are painted green, and two are painted blue. The die is rolled three times, and the colors that are rolled are recorded. Let RRG be the outcome where red is rolled the first two times, and green is rolled on the third roll.
\begin{enumerate}[label=(\alph*)]
    \item List all 27 possible outcomes.
    \begin{center}
        \begin{tabular}{ccccccccc}
        BBB & BGB & BRB & BBG & BGG & BRG & BBR & BGR & BRR \\
        GBB & GGB & GRB & GBG & GGG & GRG & GBR & GGR & GRR \\
        RBB & RGB & RRB & RBG & RGG & RRG & RBR & RGR & RRR \\
        \end{tabular}
    \end{center}

    \item Consider the event where all three rolls produce a different color. One outcome in this event is RGB. List all outcomes in the event. What is the probability of the event?
   
    Let event in which all three rolls produce a different color be denoted by $A$.

    \[A = \{\text{BGB}, \text{BRG}, \text{GRB}, \text{GBR}, \text{RGB}, \text{RBG}\}\]

    \[Pr(A) = \frac{6}{27} \approx 0.22\]

    \item Consider the event where two of the colors rolled are the same. One outcome in this event is RRB. List all outcomes in this event. What is the probability of the event?
    
    Let event in which two of the colors rolled are the same be denoted by $B$.

    \[
    B = \left\{
    \begin{array}{llllll}
    \text{BGB}, & \text{BRB}, & \text{BGG}, & \text{BRR}, \\ 
    \text{GBB}, & \text{GGB}, & \text{GRG}, & \text{GRR}, \\
    \text{RBB}, & \text{RRB}, & \text{RGG}, & \text{RRG}
    \end{array}
    \right\}
    \]

    \[Pr(B) = \frac{12}{27} \approx 0.44\]
\end{enumerate}

\item You have two hats. The first hat contains two orange marbles and seven red marbles; the second hat contains four orange marbles and three red marbles. Bob selects a marble by first choosing one of the two hats at random. He then selects one of the marbles in this hat at random. If Bob has selected a red marble, what is the probability that he selected a marble from the first hat?

Let the first and second hats be represented by $H_1$ and $H_2$, respectively. And selecting a red marble be $R$ and orange $O$.

We know the probabilities of selecting $H_1$ and $H_2$.
\[Pr(H_1) = 0.5 \text{ and } Pr(H_2) = 0.5\]

We also know the probabilities of selecting a red marble in each hat:
\[Pr(R|H_1) = \frac{7}{9} \text{ and } Pr(R|H_2) = \frac{3}{7}\]

We ask what is the probability of $H_1$ given $R$. Or $Pr(H_1 | R)$ which can be found using Bayes Theorem:

\[Pr(H_1|R) = \frac{Pr(R|H_1) \cdot Pr(H_1)}{Pr(R)}\]

We do yet know $Pr(R)$ but can find it using total probability:

\[Pr(R) = Pr(R|H_1)Pr(H_1) + Pr(R|H_2)Pr(H_2) = \frac{7}{9} \cdot 0.5 + \frac{3}{7} \cdot 0.5 = \frac{38}{63}\]

Finally to find $Pr(H_1 | R):$

\[Pr(H_1| R) = \frac{ 7/9 \cdot 0.5}{ 38 / 63} \approx 0.68\]

\item  Suppose that the words ``Act now'' occurs in 250 of 2000 messages known to be spam and in 5 of 1000 messages known not to be spam. Estimate the probability that an incoming message containing the words ``Act now'' is spam, assuming that it is initially equally likely that an incoming message is spam or not spam. If our threshold for rejecting a message as spam is 0.9, will we reject this incoming message?

Let $S$ and $\overline{S}$ be a spam and not spam message respectively. Additionally let $A$ represent if the message contains ``Act now''.

The we know the probability of spam and not spam message containing ``Act now''.

\[Pr(A|S) = \frac{250}{2000} = \frac{1}{8} \text{ and } Pr(A| \overline{S}) = \frac{5}{1000} = \frac{1}{200}\]

We also know the probability of a spam and not spam message are equal, so:
\[Pr(S) = 0.5 \text{ and } Pr(\overline{S}) = 0.5\]

We ask the probability an incoming message is spam, given it contains ``Act now'', or $Pr(S|A)$.

\[Pr(S|A) = \frac{Pr(A|S) \cdot Pr(S)}{Pr(A)}\]

We need to find $Pr(A)$ and will do this using total probability:

\[Pr(A) = Pr(A|S) \cdot Pr(S) + Pr(A|\overline{S}) \cdot Pr(\overline{S}) = \frac{1}{8} \cdot 0.5 + \frac{1}{200} \cdot 0.5 = \frac{13}{200}\]

Now can find $Pr(S|A)$:
\[Pr(S|A) = \frac{1/8 \cdot 0.5}{13/200} \approx 0.96\]

We will reject the incoming message.

\end{enumerate}


\end{document}