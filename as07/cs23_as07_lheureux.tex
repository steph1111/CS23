\documentclass[11pt, letterpaper, includehead]{article}

%%%%%%%%%%%%%%%%%%%%% Pre-document %%%%%%%%%%%%%%%%%%%%%
\usepackage{fancyhdr}
\usepackage{float}
\usepackage{array}
\usepackage{nicematrix}
\usepackage{enumitem}
\usepackage{titlesec}
\usepackage{multicol}
\usepackage{scrextend}
\usepackage{hyperref}
\usepackage{amssymb}
\usepackage{amsthm}

\setlength{\parindent}{0pt} % Remove auto paragraph indents

% Get rid of those big margins
\usepackage[margin=1in]{geometry}
\newlength\titleindent
\setlength\titleindent{2cm}

\makeatletter
\def\@mathmargin{1in}
\makeatother

\titleformat{\section}[runin]
    {\normalfont\bfseries}% formatting commands to apply to the whole heading
    {\thesubsection}% the label and number
    {0.5em}% space between label/number and subsection title
    {}% formatting commands applied just to subsection title
    []% punctuation or other commands following subsection title

\theoremstyle{plain}

\newtheoremstyle{mydefinition}  % Name of the style
  {} % Space above
  {} % Space below
  {\normalfont} % Body font (normal, not bold or italic)
  {} % Indent amount
  {\itshape} % Theorem head font (italic)
  {.} % Punctuation after theorem head
  {.5em} % Space after theorem head
  {} % Theorem head spec (can be left empty)

\theoremstyle{mydefinition}
\newtheorem{defin}{Definition}

\newtheoremstyle{myproperty}  % Name of the style
  {} % Space above
  {} % Space below
  {\normalfont} % Body font (normal, not bold or italic)
  {} % Indent amount
  {\itshape} % Theorem head font (italic)
  {.} % Punctuation after theorem head
  {.5em} % Space after theorem head
  {} % Theorem head spec (can be left empty)

\theoremstyle{myproperty}
\newtheorem{prop}{Property}

\begin{document} 

\pagestyle{fancy}
% Header
\fancyhead{}
\fancyhead[L]{\textbf{CS23:} Assignment \#7}
\fancyhead[R]{\href{mailto:stepanielh1111@gmail.com}{L'Heureux} \thepage}
% No page numbers for footer
\fancyfoot{}


\begin{center}
    \Large{\textbf{Assignment 7}}\\
    \Large{Counting: Part 1}
\end{center}

\begin{enumerate}[label=\textbf{\arabic*}., leftmargin=*]
\item For your college interview, you must wear a tie. You own 3 regular (boring) ties and 5 (cool) bow ties.
\begin{enumerate}[label=(\alph*)]
    \item How many choices do you have for your neck-wear?

    Event $A$: 3 ties \\
    Event $B$: 5 bow ties

    \[
    3 + 5 = 8 \quad \text{(A or B)}
    \]

    Therefore, you have 8 choices for your neck-wear.

    \item You realize that the interview is for clown college, so you should probably wear both a regular tie and a bow tie. How many choices do you have now?
    
    Event $A$: 3 ties \\
    Event $B$: 5 bow ties

    \[
    3 \cdot 5 = 15 \quad \text{(A and B)}
    \]
    Then, you have 15 choices for your neck-wear.
    \item For the rest of your outfit, you have 5 shirts, 4 skirts, 3 pants, and 7 dresses. You want to select either a shirt to wear with a skirt or pants, or just a dress. How many outfits do you have to choose from?

    Event $A$: 5 shirts \\
    Event $B$: 4 skirts \\
    Event $C$: 3 pants \\
    Event $D$: 7 dresses

    \[
    5 \cdot 4 + 5 \cdot 3 + 7 =  42\quad \text{(A and B or A and C or D)}
    \]

    Therefore, you have 42 choices for your outfit.
\end{enumerate}

\item Hexadecimal, or base 16, uses 16 distinct digits that can be used to form numbers: {0,1,...,9,A,B,C,D,E,F}. So for example, a 3 digit hexadecimal number might be 2B8.
\begin{enumerate}[label=(\alph*)]
    \item How many 2-digit hexadecimals are there in which the first digit is E or F? Explain your answer in terms of the additive principle (using either events or sets).

    There are 16 possibilities for 2-digit hexadecimals starting with E and 16 possibilities for 2-digit hexadecimals starting with F. Let Event A and Event B represent these situations, respectively.

    We then have:

    \[
    16 + 16 = 32
    \]

    There are 32 such hexadecimals.
    \item Explain why your answer to the previous part is correct in terms of the multiplicative principle (using either events or sets). Why do both the additive and multiplicative principles give you the same answer?

    We may also assess this problem in terms of the Multiplicative Principal. We restrict the first digit to E or F. The second digit is not restricted and then has 16 possibilities. For each of the 2 choices of leading digit, we have 16 choices for the ending digit.

    We then have:

    \[
    2 \cdot 16 = 32
    \]

    This is the same as the answer given in part (a) as the multiplicative principal allows us to generalize the additive principal for events of the same size.

    \item How many 3-digit hexadecimals start with a letter (A-F) and end with a numeral (0-9)? Explain.
    
    We have 6 choices for the leading digit, 10 choices for the ending digit, and since the middle digit is not restricted, it has 16 choices.

    Event $A$: 6 choices for the leading digit\\
    Event $B$: 16 choices for the middle digit\\
    Event $C$: 10 choices for the ending digit 


    We then have:
    \[
    6 \cdot 16 \cdot 10 = 960 \quad \text{(A and B and C)}
    \]
    
    \item How many 3-digit hexadecimals start with a letter (A-F) or end with a numeral (0-9) (or both)? Explain.
   
    We have three digits: 
    \[
        \underline{D_2} \; \underline{D_1} \; \underline{D_0}
    \]


    And three cases:

    $C_1$: $D_2$ is a letter and $D_1$ and $D_0$ are unrestricted.

    $C_2$: $D_0$ is a number and $D_2$ and $D_1$ are unrestricted.

    $C_3$: $D_2$ is a letter and $D_0$ is a number and $D_1$ is unrestricted.

    \begin{align*}
        C_1: & \quad \underline{6} \text{ and } \underline{16} \text{ and } \underline{16} & 6 \cdot 16 \cdot 16 = 1536 \text{ possibilities}\\
        C_2: & \quad \underline{16} \text{ and } \underline{16} \text{ and } \underline{10} & 16 \cdot 16 \cdot 10 = 2560 \text{ possibilities}\\
        C_3: & \quad \underline{6} \text{ and } \underline{16} \text{ and } \underline{10} & 6 \cdot 16 \cdot 10 = 960 \text{ possibilities}
    \end{align*}

    The cases are not disjoint so we must remove the duplicates which are given by $C_3$
    Then:
    \[1536 + 2560 - 960 = 3136\]


\end{enumerate}

\item If $|M|$ 100 and $|N| = 42$, what is $|M \cup N| + |M \cap N|$?
\begin{align*}
    |M \cup N| + |M \cap N| &= |A| + |B| - |A \cap B| + |M \cap N| \\
    |M \cup N| + |M \cap N| &= |A| + |B| \\
    \intertext{If $|M| = 100$ and $|N| = 42$, then:}
    |M \cup N| + |M \cap N| &= 100 + 42 \\
    |M \cup N| + |M \cap N| &= 142
\end{align*}

\item Consider all 5 letter ``words'' made from the letters a through f. (Recall, words are just strings of letters, not necessarily actual English words.)
\begin{enumerate}[label=(\alph*)]
    \item How many of these words are there total?
    \[
    6^5 = 7776
    \]
    There are 7776 words in total.
    \item How many of these words contain no repeated letters?
    \[
    6 \cdot 5 \cdot 4 \cdot 3 \cdot 2 = 720
    \] 
    There are 720 words in total.

    \item How many of these words start with the sub-word ``aba''?
    
    When the first 3 letters are restricted to be the sub-word ``aba'', then we are free to choose the last two letters:
    \[
    6^2 = 36
    \]
    \item How many of these words either start with ``abc'' or end with ``cba'' or both?
    
    Start with ``abc'':
    \[
    6^2 = 36
    \]
    End with ``cab'':
    \[
    6^2 = 36
    \]
    There is one case of overlap when a word starts with ``abc'' and ends with ``cba'' ``abcba''.

    So the total number of words which satisfy the condition is $36 + 36 - 1 = 71$

    \item How many of the words containing no repeats also do not contain the sub-word ``bad''?
    
    If bad is a sub-word, then there are two positions remaining in the word for which we may choose letters.

    ``bad'' contains the letters `a', `b', `d', so to avoid repeats, the letters we may choose from are `c', `e', `f'.

    There are three arrangements.

    Then the number of words which have no repeats and contain the sub-word ``bad'' is then given by:
    \[2 \cdot 3 \cdot 3 = 18\]

    Then the words which do not contain the sub-word ``bad'' are:

    \[720 - 16 = 702\]
    
\end{enumerate}

\item Consider the bit strings in $B(6,2)$ (bit strings of length 6 and weight 2).
\begin{enumerate}[label=(\alph*)]
    \item How many of those bit strings start with 1?
    
    If the leading bit is restricted to 1, the other 5 digits form the bit string $B(5, 1)$.
    
    \[|\mathbf{B}_1^5| = {5 \choose 1} = 5 \text{ bit strings}\]
    \item How many of those bit strings start with 01?
    
    If the leading bits are restricted to be 01, the other 4 digits form the bit string $B(4, 1)$.
    \[|\mathbf{B}_1^4| = {4 \choose 1} = 4 \text{ bit strings}\]
    
    \item How many of those bit strings start with 001?
    
    If the leading bits are restricted to be 001, the other 3 digits form the bit string $B(3, 1)$.
    \[|\mathbf{B}_1^3| = {3 \choose 1} = 3 \text{ bit strings}\]

    \item Are there any other strings we have not counted yet? Which ones, and how many are there?
    
    We have not counted strings which start with 0001 and 0000
    
    If the leading bits are restricted to be 0001, the other 2 digits form the bit string $B(2, 1)$.
    \[|\mathbf{B}_1^2| = {2 \choose 1} = 2 \text{ bit strings}\]

    If the leading bits are restricted to be 0000, the other 2 digits form the bit string $B(2, 2)$.
    \[|\mathbf{B}_2^2| = {2 \choose 2} = 1 \text{ bit string}\]

    So there were 3 more bit strings we did not previously count.

    \item How many bit strings are there total in $B(6,2)$?

    \[|\mathbf{B}_2^6| = {6 \choose 2} = 15 \text{ bit strings}\]
\end{enumerate}
\end{enumerate}


\end{document}