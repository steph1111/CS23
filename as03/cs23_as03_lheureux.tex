\documentclass[11pt, letterpaper, includehead]{article}

%%%%%%%%%%%%%%%%%%%%% Pre-document %%%%%%%%%%%%%%%%%%%%%
\usepackage{fancyhdr}
\usepackage{float}
\usepackage{array}
\usepackage{nicematrix}
\usepackage{enumitem}
\usepackage{titlesec}
\usepackage{multicol}
\usepackage{scrextend}
\usepackage{hyperref}
\usepackage{amssymb}
\usepackage{amsthm}

\setlength{\parindent}{0pt} % Remove auto paragraph indents

% Get rid of those big margins
\usepackage[margin=1in]{geometry}
\newlength\titleindent
\setlength\titleindent{2cm}

\makeatletter
\def\@mathmargin{1in}
\makeatother

\titleformat{\section}[runin]
    {\normalfont\bfseries}% formatting commands to apply to the whole heading
    {\thesubsection}% the label and number
    {0.5em}% space between label/number and subsection title
    {}% formatting commands applied just to subsection title
    []% punctuation or other commands following subsection title

\theoremstyle{plain}

\newtheoremstyle{mydefinition}  % Name of the style
  {} % Space above
  {} % Space below
  {\normalfont} % Body font (normal, not bold or italic)
  {} % Indent amount
  {\itshape} % Theorem head font (italic)
  {.} % Punctuation after theorem head
  {.5em} % Space after theorem head
  {} % Theorem head spec (can be left empty)

\theoremstyle{mydefinition}
\newtheorem{defin}{Definition}

\newtheoremstyle{myproperty}  % Name of the style
  {} % Space above
  {} % Space below
  {\normalfont} % Body font (normal, not bold or italic)
  {} % Indent amount
  {\itshape} % Theorem head font (italic)
  {.} % Punctuation after theorem head
  {.5em} % Space after theorem head
  {} % Theorem head spec (can be left empty)

\theoremstyle{myproperty}
\newtheorem{prop}{Property}

\begin{document} 

\pagestyle{fancy}
% Header
\fancyhead{}
\fancyhead[L]{\textbf{CS23:} Assignment \#3}
\fancyhead[R]{\href{mailto:stepanielh1111@gmail.com}{L'Heureux} \thepage}
% No page numbers for footer
\fancyfoot{}


\begin{center}
    \Large{\textbf{Assignment 3}}\\
    \Large{Direct Proofs}
\end{center}

\begin{enumerate}[label=\textbf{\arabic*}., leftmargin=*]
\item Prove of Disprove: Every positive integer can be expressed as a sum of three or fewer
    perfect squares.
\begin{defin}\label{def:def1a}
    An integer $n$ is called a perfect square if, and only if, $n = k^2$ for some integer $k$.
\end{defin}
\begin{proof}
    The theorem may be restated using the definition of perfect squares\ref{def:def1a}:
    \[m = a^2 + b^2 + c^2, \quad \text{for some integers a, b, c, m.}\]

    Let $m = 7$:
    \[7 = a^2 + b^2 + c^2\]

    The square of any integer is positive, and the sum of positive integers is always greater than each individual term of the sum. Therefore, $a, b, c \in [0, 6]$.

    Testing all cases:

    \[7 \neq 0^2 + 0^2 + 0^2 = 0, \quad a = 0, b = 0, c = 0\]
    \[7 \neq 1^2 + 0^2 + 0^2 = 1, \quad a = 1, b = 0, c = 0\]
    \[7 \neq 1^2 + 1^2 + 0^2 = 2, \quad a = 1, b = 1, c = 0\]
    \[7 \neq 1^2 + 1^2 + 1^2 = 3, \quad a = 1, b = 1, c = 1\]
    \[7 \neq 2^2 + 0^2 + 0^2 = 4, \quad a = 2, b = 0, c = 0\]
    \[7 \neq 2^2 + 1^2 + 0^2 = 5, \quad a = 2, b = 1, c = 0\]
    \[7 \neq 2^2 + 1^2 + 1^2 = 6, \quad a = 2, b = 1, c = 1\]
    \[7 \neq 2^2 + 2^2 + 0^2 = 8, \quad a = 2, b = 2, c = 0\]

    There are no integers $a, b, c$ such that $7 = a^2 + b^2 + c^2$. Therefore every positive integer cannot be expressed as a sum of three or fewer
    perfect squares.

\end{proof}

\pagebreak 

\item Prove or Disprove: If $m$ is any even integer and $n$ is any odd integer,
then $(m + 2)^2 - (n - 1)^2$ is even.
\begin{prop}\label{prop:prop2a}
    The sum, product, or difference of any two even integers is even.
\end{prop}

\begin{prop}\label{prop:prop2b}
    The sum, or difference of any two odd integers is even.
\end{prop}
\begin{proof} 
    $m$ is even, so it can be described as:
    \[ m = 2a, \quad \text{for some integer } a.\]
    $n$ is odd, so it can be described as:
    \[n = 2b - 1, \quad \text{for some integer } b.\]

    Evaluating the first term:

    \[(m + 2) = (2a + 2) = 2(a + 1), \quad \text{which is even}\]

    Evaluating the second term:
    \[(n - 1) = ((2b - 1) - 1) = (2b - 2) = 2(b - 1), \quad \text{which is even}\]

    $(m + 2)$ and $(n - 1)$ are both even numbers.
    
    $(m + 2)(m + 2)$ and $(n - 1)(n - 1)$ are the product of two even numbers, which by property \ref{prop:prop2a} is also even.

    Furthermore, by \ref{prop:prop2a}, the difference of two even numbers, $(m + 2)^2 - (n - 1)^2$, is even.

    Therefore, if $m$ is any even integer and $n$ is any odd integer, then $(m + 2)^2 - (n - 1)^2$ is even.
\end{proof}


\end{enumerate}

\end{document}