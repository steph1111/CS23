\documentclass[11pt, letterpaper, includehead]{article}

%%%%%%%%%%%%%%%%%%%%% Pre-document %%%%%%%%%%%%%%%%%%%%%
\usepackage{fancyhdr}
\usepackage{float}
\usepackage{array}
\usepackage{nicematrix}
\usepackage{enumitem}
\usepackage{titlesec}
\usepackage{multicol}
\usepackage{scrextend}
\usepackage{hyperref}
\usepackage{amssymb}
\usepackage{amsthm}
\usepackage{tikz}

\setlength{\parindent}{0pt} % Remove auto paragraph indents

% Get rid of those big margins
\usepackage[margin=1in]{geometry}
\newlength\titleindent
\setlength\titleindent{2cm}

\makeatletter
\def\@mathmargin{1in}
\makeatother

\titleformat{\section}[runin]
{\normalfont\bfseries}% formatting commands to apply to the whole heading
{\thesubsection}% the label and number
{0.5em}% space between label/number and subsection title
{}% formatting commands applied just to subsection title
[]% punctuation or other commands following subsection title

\theoremstyle{plain}

\newtheoremstyle{mydefinition}	% Name of the style
{} % Space above
{} % Space below
{\normalfont} % Body font (normal, not bold or italic)
{} % Indent amount
{\itshape} % Theorem head font (italic)
{.} % Punctuation after theorem head
{.5em} % Space after theorem head
{} % Theorem head spec (can be left empty)

\theoremstyle{mydefinition}
\newtheorem{defin}{Definition}

\newtheoremstyle{myproperty}  % Name of the style
{} % Space above
{} % Space below
{\normalfont} % Body font (normal, not bold or italic)
{} % Indent amount
{\itshape} % Theorem head font (italic)
{.} % Punctuation after theorem head
{.5em} % Space after theorem head
{} % Theorem head spec (can be left empty)

\theoremstyle{myproperty}
\newtheorem{prop}{Property}

\begin{document}

\pagestyle{fancy}
% Header
\fancyhead{}
\fancyhead[L]{\textbf{CS23:} Assignment \#5}
\fancyhead[R]{\href{mailto:stepanielh1111@gmail.com}{L'Heureux} \thepage}
% No page numbers for footer
\fancyfoot{}

\begin{center}
    \Large{\textbf{Assignment 5}}\\
    \Large{Trees and More Graphs}
\end{center}

\begin{enumerate}[label=\textbf{\arabic*}., leftmargin=*]
    \item We often define graph theory concepts using set theory. For example given a graph $G = (V, E)$ and a vertex $v \in V$, we define
    \[N(v) = \{ u \in V : {v, u} \in E\} \]
    We define $N[v] = N(v) \cup \{ v\} $. The goal of this problem is to figure out what all this means.

        
    \begin{enumerate}[label=(\alph*)]
        \item Let $G$ be the graph with $V = \{ a,b,c,d,e,f \} $ and
            \\$E = \{  \{ a,b \}, \{ a,e \},\{ b,c \}, \{ b,e \}, \{ c, d\}, \{ c, f\}, \{ d,f \}, \{ e,f \} \} $. 
            Find $N(a), N[a], N(c), \text{and} N[c]$.
        \item What is the largest and smallest possible values for $|N(v)|$ and $|N[v]|$ for the graph from part (a)? Explain.
        \item Give an example of a graph $G = (V, E)$ (Probably different from the one above) for which $N[v] = V$ for some vertex $v \in V$. Is there a graph for which $N[v] = V$ for \emph{all} $v \in V$? Explain.
        \item Give an example of a graph $G = (V, E)$ for which $N(v) = \emptyset$  for some $v \in V$. Is there an example of such graph for which $N[u] = V$ for some other $u \in V$ as well? Explain.
        \item Describe in words what $N(v)$ and $N[v]$ mean in general.
    \end{enumerate}

    \item Which of the following graphs are trees
    \begin{enumerate}[label=(\alph*)]
        \item $G = (V, E)$ with $V = \{a, b, c, d, e\}$ and $E = \{ \{a, b \}, \{a, e \}, \{b, c \}, \{c, d \}, \{d, e\}\}$
        \item $G = (V, E)$ with $V = \{a, b, c, d, e\}$ and $E = \{ \{a, b \}, \{b, c \}, \{c, d \}, \{d, e \}\}$
        \item $G = (V, E)$ with $V = \{a, b, c, d, e\}$ and $E = \{ \{a, b \}, \{a, c \}, \{a, d \}, \{a, e \}\}$
        \item $G = (V, E)$ with $V = \{a, b, c, d, e\}$ and $E = \{ \{a, b \}, \{a, c \}, \{d, e \}\}$
    \end{enumerate}

    \item For each degree sequence below, decide wether it must always, must never, or could possibly be a degree sequence for a tree. Remember, a degree sequence lists out the degrees (number of edges incident to the vertex) of all the vertices in a graph in non-increasing order.
    \begin{enumerate}[label=(\alph*)]
        \item $(4, 1, 1, 1, 1)$
        \item $(3, 3, 2, 1, 1)$
        \item $(2, 2, 2, 1, 1)$
        \item $(4, 4, 3, 3, 3, 2, 2, 1, 1, 1, 1, 1, 1, 1)$
    \end{enumerate}

    \item Suppose you have a graph with $v$ vertices and $e$ edges that satisfies $v = e + 1$. Must the graph be a tree? Prove your answer.
    \item Prove that any graph (not necessarily a tree) with $v$ vertices and $e$ edges that satisfies $v > e + 1$ will NOT be connected.
    \item Let $T$ be a rooted tree that contains vertices $v, u, \text{and} w$ (among possibly others). Prove that if $w$ is a descendant of both $u \text{and} v$ then $u$ is a descendant of $v$ or $v$ is a descendant of $u$.
    \item Prove that every connected graph which is not itself a tree must have at last three different spanning trees.
\end{enumerate}

\end{document}